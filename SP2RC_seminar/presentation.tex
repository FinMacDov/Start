%%%%%%%%%%%%%%%%%%%%%%%%%%%%%%%%%%%%%%%%%
% Beamer Presentation
% LaTeX Template
% Version 1.0 (10/11/12)
%
% This template has been downloaded from:
% http://www.LaTeXTemplates.com
%
% License:
% CC BY-NC-SA 3.0 (http://creativecommons.org/licenses/by-nc-sa/3.0/)
%
%%%%%%%%%%%%%%%%%%%%%%%%%%%%%%%%%%%%%%%%%

%----------------------------------------------------------------------------------------
%	PACKAGES AND THEMES
%----------------------------------------------------------------------------------------

\documentclass{beamer}

\mode<presentation> {

% The Beamer class comes with a number of default slide themes
% which change the colors and layouts of slides. Below this is a list
% of all the themes, uncomment each in turn to see what they look like.

%\usetheme{default}
%\usetheme{AnnArbor}
%\usetheme{Antibes}
%\usetheme{Bergen}
%\usetheme{Berkeley}
%\usetheme{Berlin}
%\usetheme{Boadilla}
%\usetheme{CambridgeUS}
%\usetheme{Copenhagen}
%\usetheme{Darmstadt}
%\usetheme{Dresden}
%\usetheme{Frankfurt}
%\usetheme{Goettingen}
%\usetheme{Hannover}
%\usetheme{Ilmenau}
%\usetheme{JuanLesPins}
%\usetheme{Luebeck}
\usetheme{Madrid}
%\usetheme{Malmoe}
%\usetheme{Marburg}
%\usetheme{Montpellier}
%\usetheme{PaloAlto}
%\usetheme{Pittsburgh}
%\usetheme{Rochester}
%\usetheme{Singapore}
%\usetheme{Szeged}
%\usetheme{Warsaw}

% As well as themes, the Beamer class has a number of color themes
% for any slide theme. Uncomment each of these in turn to see how it
% changes the colors of your current slide theme.

%\usecolortheme{albatross}
%\usecolortheme{beaver}
%\usecolortheme{beetle}
%\usecolortheme{crane}
%\usecolortheme{dolphin}
%\usecolortheme{dove}
%\usecolortheme{fly}
%\usecolortheme{lily}
%\usecolortheme{orchid}
%\usecolortheme{rose}
%\usecolortheme{seagull}
%\usecolortheme{seahorse}
%\usecolortheme{whale}
%\usecolortheme{wolverine}

%\setbeamertemplate{footline} % To remove the footer line in all slides uncomment this line
%\setbeamertemplate{footline}[page number] % To replace the footer line in all slides with a simple slide count uncomment this line

%\setbeamertemplate{navigation symbols}{} % To remove the navigation symbols from the bottom of all slides uncomment this line
}

\usepackage{graphicx} % Allows including images
\usepackage{booktabs} % Allows the use of \toprule, \midrule and \bottomrule in tables

%----------------------------------------------------------------------------------------
%	TITLE PAGE
%----------------------------------------------------------------------------------------

\title[ ]{MHD Simulations of Jets with Applications to the Sun} % The short title appears at the bottom of every slide, the full title is only on the title page

\author[Fionnlagh Mackenzie Dover]{Fionnlagh Mackenzie Dover \\ Supervisor: Prof R\'{o}bert Erd\'{e}lyi } % Your name
\institute[SP$^2$RC] % Your institution as it will appear on the bottom of every slide, may be shorthand to save space
{
University of Sheffield \\ % Your institution for the title page
\medskip
}
\date{03/11/2017} % Date, can be changed to a custom date

\begin{document}

\begin{frame}
\titlepage % Print the title page as the first slide
\end{frame}

\begin{frame}{Overview}

\tableofcontents % Throughout your presentation, if you choose to use \section{} and \subsection{} commands, these will automatically be printed on this slide as an overview of your presentation
\end{frame}

%----------------------------------------------------------------------------------------
%	PRESENTATION SLIDES
%----------------------------------------------------------------------------------------

%------------------------------------------------
\section{Intoduction} % Sections can be created in order to organize your presentation into discrete blocks, all sections and subsections are automatically printed in the table of contents as an overview of the talk
%------------------------------------------------
\subsection{Cornal Heating}
\begin{frame}
Test
\end{frame}

\subsection{Transition Region Quakes} % A subsection can be created just before a set of slides with a common theme to further break down your presentation into chunks
\begin{frame}

\end{frame}
\subsection{Scientific Goal}
\begin{frame}

\end{frame}
\section{MPI-AMRVAC}
\subsection{Overview}
\begin{frame}
\begin{block}{History}
\begin{itemize}
\item Talk a bit of VAC.
\item Purpose of the code
\item Strucutre of the code
\end{itemize}
\end{block}
\end{frame}
%------------------------------------------
\subsection{Adavitive Mesh Refienment}
\begin{frame}
\begin{block}{AMR}
\begin{itemize}
\item diagrams to explain
\item example by movie
\end{itemize}
\end{block}
\end{frame}
%-------------------------------------------------------
\begin{frame}
\begin{block}{Local Error Estimation}
Outline AMR criteria.
\end{block}
\end{frame}
\subsection{MHD Module}
\begin{frame}
\begin{block}{MHD Module}
Have equations.
\end{block}
\end{frame}

%------------------------------------------------


%----------------------------------------------------------------------------------------

\end{document}
